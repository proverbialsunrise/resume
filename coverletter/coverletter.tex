%%te%%%%%%%%%%%%%%%%%%%%%%%%%%%%%%%%%%%%%%%
% "ModernCV" CV and Cover Letter
% LaTeX Template
% Version 1.1 (9/12/12)
%
% This template has been downloaded from:
% http://www.LaTeXTemplates.com
%
% Original author:
% Xavier Danaux (xdanaux@gmail.com)
%
% License:
% CC BY-NC-SA 3.0 (http://creativecommons.org/licenses/by-nc-sa/3.0/)
%
% Important note:
% This template requires the moderncv.cls and .sty files to be in the same 
% directory as this .tex file. These files provide the resume style and themes 
% used for structuring the document.
%
%%%%%%%%%%%%%%%%%%%%%%%%%%%%%%%%%%%%%%%%%

%----------------------------------------------------------------------------------------
%	PACKAGES AND OTHER DOCUMENT CONFIGURATIONS
%----------------------------------------------------------------------------------------

\documentclass[11pt,a4paper,sans]{moderncv} % Font sizes: 10, 11, or 12; paper sizes: a4paper, letterpaper, a5paper, legalpaper, executivepaper or landscape; font families: sans or roman

\moderncvstyle{casual} % CV theme - options include: 'casual' (default), 'classic', 'oldstyle' and 'banking'
\moderncvcolor{blue} % CV color - options include: 'blue' (default), 'orange', 'green', 'red', 'purple', 'grey' and 'black'


\usepackage[scale=0.75]{geometry} % Reduce document margins
%\setlength{\hintscolumnwidth}{3cm} % Uncomment to change the width of the dates column
%\setlength{\makecvtitlenamewidth}{10cm} % For the 'classic' style, uncomment to adjust the width of the space allocated to your name

%----------------------------------------------------------------------------------------
%	NAME AND CONTACT INFORMATION SECTION
%----------------------------------------------------------------------------------------

\firstname{Daniel} % Your first name
\familyname{Johnson} % Your last name

% All information in this block is optional, comment out any lines you don't need
%\title{Curriculum Vitae}
\address{142 Wood St. Apt. B}{Kitchener, ON N2G2H8}
\mobile{(226) 989 0511}
%\phone{(000) 111 1112}
%\fax{(000) 111 1113}
\email{dan@ddajohnson.com}
\homepage{http://ddajohnson.com}{ddajohnson.com} % The first argument is the url for the clickable link, the second argument is the url displayed in the template - this allows special characters to be displayed such as the tilde in this example
%\extrainfo{additional information}
%\photo[70pt][0.4pt]{picture} % The first bracket is the picture height, the second is the thickness of the frame around the picture (0pt for no frame)
%\quote{"A witty and playful quotation" - John Smith}

%----------------------------------------------------------------------------------------

\begin{document}


\clearpage

\recipient{Clearpath Robotics}{1425 Strasburg Rd. Suite 2A \\ Kitchener, ON N2R 1H2} % Letter recipient
\date{\today} % Letter date
\opening{Dear Hiring Manager,} % Opening greeting
\closing{Sincerely yours,} % Closing phrase

\makelettertitle % Print letter title

I am an engineer interested in working at Clearpath Robotics. I'm excited about the possibilities for using autonomous robots to replace humans in dangerous or dull workplaces and I believe that Clearpath will be an important part of that revolution. Reading about your commitment to open source projects and knowing first-hand about your cooperation with academic projects, I believe that Clearpath is taking the right approach to revolutionizing robotics. Along with my admiration for your strategy, I have the technical skills to excel in a role as an Autonomy Engineer.

For the past two years, I've been a graduate student at the University of Waterloo building 3-D simulations of the golf swing for use in designing golf clubs. By building a model of the golfer and club using MapleSim and MATLAB on Linux, I was able to evaluate golf club design decisions without building or experimenting with physical golf clubs. To control the model, I took courses in multi-variate, adaptive, and optimal control to find a way to optimize the swing for different parameter sets. The final model was delivered along with concise documentation to an outside partner to be used as part of their evaluation process for new golf clubs.

As my fourth-year design project, I used ROS and Gazebo to create a simulation of an autonomous robot for the \href{http://www.nasa.gov/directorates/spacetech/centennial_challenges/sample_return_robot/index.html}{NASA Sample Return Robot Challenge}. My group's responsibility was to develop a Simultaneous Location and Mapping algorithm for a 6 degree of freedom environment without using GPS sensors. By combining wheel odometry, IMU measurements, and laser scan point clouds in a novel way, we were able to improve on existing techniques for locating the robot in its environment. The final algorithm incorporated a Kalman Filter, the Iterative Closest Point algorithm for aligning point clouds, and GraphSLAM for redistributing error in the estimated positions. A Clearpath Husky robot was used as our platform for this competition and since that time, I've always thought of Clearpath as a place I would like to work.

I had many opportunities as part of the coop program at the University of Waterloo to demonstrate my skills in development.  At Apple Inc. I designed and developed an internal application for testing location algorithms that included iOS, OS X, and server-side components using Objective-C, C++, and Python. This application was used to help develop a new Kalman Filter based algorithm for determining device locations. In the Vision and Image Processing Lab at uWaterloo I worked on \href{http://vip.uwaterloo.ca/demos/satellite-sar-sea-ice-classification}{MAGIC}, a computer vision program written in C++ used for segmenting satellite images of sea-ice. I developed a new algorithm for this product that used a multi-level segmentation to classify large images.  On my own, I've built a few smaller projects which can be found on my website at \url{http://ddajohnson.com/projects}.

I'm excited by the chance to work at Clearpath in a dynamic environment solving interesting problems with autonomous robots. I hope to hear from you soon.

\makeletterclosing % Print letter signature

%----------------------------------------------------------------------------------------

\end{document}
